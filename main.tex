\makeatletter%cmepda
\let\@starttocorig\@starttoc
\makeatother%%

\documentclass[t,10pt,xcolor={usenames},fleqn]{beamer}

%%%Usefull link
%tikz-equations:
%http://www.wekaleamstudios.co.uk/posts/creating-a-presentation-with-latex-beamer-equations-and-tikz/

\hypersetup{pdfpagemode=FullScreen}

%% colors
\definecolor{NOW}{rgb}{1.0, 0.75, 0.0}
\definecolor{suspendend}{rgb}{0.76, 0.6, 0.42}
\definecolor{bittersweet}{rgb}{1.0, 0.44, 0.37}
\definecolor{brilliantlavender}{rgb}{0.96, 0.73, 1.0}
\definecolor{antiquefuchsia}{rgb}{0.57, 0.36, 0.51}
\definecolor{violetw}{rgb}{0.93, 0.51, 0.93}
\definecolor{Veronica}{rgb}{0.63, 0.36, 0.94}
\definecolor{atomictangerine}{rgb}{1.0, 0.6, 0.4}
\definecolor{darkgray}{rgb}{0.66, 0.66, 0.66}
\definecolor{brightcerulean}{rgb}{0.11, 0.67, 0.84}
\definecolor{cadmiumorange}{rgb}{0.93, 0.53, 0.18}
\definecolor{ochre}{rgb}{0.8, 0.47, 0.13}
\definecolor{midnightblue}{rgb}{0.1, 0.1, 0.44}
\definecolor{lemon}{rgb}{1.0, 0.97, 0.0}
\definecolor{grey}{rgb}{0.7, 0.75, 0.71}
\definecolor{amber}{rgb}{1.0, 0.75, 0.0}
\definecolor{almond}{rgb}{0.94, 0.87, 0.8}
\definecolor{bf}{RGB}{88, 86, 88}
\definecolor{bb}{RGB}{177, 177, 177}
\definecolor{keyword}{rgb}{0.25, 0.25, 0.28}
\definecolor{coolgrey}{rgb}{0.55, 0.57, 0.67}
\definecolor{modulo}{rgb}{0.8, 0.0, 0.0}

%%%%%%%%%%%%%%%%%%%%%%%%%%%%%%%%%%% importa pacchetti
\usepackage{usepkg}
%%
\renewbibmacro*{cite}{%
  \iffieldundef{shorthand}
    {\ifnameundef{labelname}
       {}
       {\printnames{labelname}%
        \setunit{\printdelim{nametitledelim}}}%
     \usebibmacro{cite:title}}%
    {\usebibmacro{cite:shorthand}}%
  \usebibmacro{cite:url}}

\newbibmacro{cite:url}{%
  \ifentrytype{online}
    {\setunit{\addspace}%
    \printfield{url}}
    {}}
    

%%%%%%%%%%%%%%%%%%%%%%%%%%%%%%%%%%% Funzioni generali
\usepackage{functions}
%http://tex.stackexchange.com/questions/246/when-should-i-use-input-vs-include
\newcommand{\setmuskip}[2]{#1=#2\relax} %%problem usinig mu with calc (req by mathtools) loaded
\usepackage{sources}
%\usepackage{length}
%%%%%%%%%%%%%%%%%%%%%%%%%%%%%%%%%%% Funzioni per questo file main
\usepackage{mathOp}
\usepackage{beamersetup}

\def\status{coazione}%ripetere
\def\keeptrying{coazione}
\usepackage{LocalF}
%%%%%%%%%%%%%%%%%%%%%%%%%%%%%%%%%

\title{Metodi Computazionali (Python, Parallel computing, Machine learning): Febbraio 2020.}

% Let's get started
\begin{document}
%\let\mybibcat\noexpand\currfilebase
\tikzset{myscalar/.style={%%scale plot even cs axis: \def\myscale{}
		node distance=\myscale cm and \myscale cm,
		every node/.style={scale=\myscale},
		every axis/.style={
			major tick length=0.15*\myscale cm,
			minor tick length=0.1*\myscale cm,
			mark options={scale=\myscale},
			scale=\myscale
		}
	}
}
\begin{frame}
  \titlepage
\end{frame}

\begin{frame}{TOC}
\tableofcontents[onlyparts]
\end{frame}
% Section and subsections will appear in the presentation overview
% and table of contents.
%\frame{\tableofcontents[onlyparts]}

\part{Intro}\linkdest{intro}
\documentclass[main.tex]{subfiles}
 
\begin{document}

\section{Progetti scientifico-computazionali}

\begin{frame}{Brainstorming}
\begin{itemize}
    \item Visibility and field simulation
    \item Exoplanets - Histo plots, 
\end{itemize}
\end{frame}


\end{document}

\part{Succo}\linkdest{succo}
\documentclass[main.tex]{subfiles}
 
\begin{document}

\section{Binary computation}

\end{document}




\end{document}